\documentclass[11pt,preprint, authoryear]{elsarticle}

\usepackage{lmodern}
%%%% My spacing
\usepackage{setspace}
\setstretch{1.2}
\DeclareMathSizes{12}{14}{10}{10}

% Wrap around which gives all figures included the [H] command, or places it "here". This can be tedious to code in Rmarkdown.
\usepackage{float}
\let\origfigure\figure
\let\endorigfigure\endfigure
\renewenvironment{figure}[1][2] {
    \expandafter\origfigure\expandafter[H]
} {
    \endorigfigure
}

\let\origtable\table
\let\endorigtable\endtable
\renewenvironment{table}[1][2] {
    \expandafter\origtable\expandafter[H]
} {
    \endorigtable
}


\usepackage{ifxetex,ifluatex}
\usepackage{fixltx2e} % provides \textsubscript
\ifnum 0\ifxetex 1\fi\ifluatex 1\fi=0 % if pdftex
  \usepackage[T1]{fontenc}
  \usepackage[utf8]{inputenc}
\else % if luatex or xelatex
  \ifxetex
    \usepackage{mathspec}
    \usepackage{xltxtra,xunicode}
  \else
    \usepackage{fontspec}
  \fi
  \defaultfontfeatures{Mapping=tex-text,Scale=MatchLowercase}
  \newcommand{\euro}{€}
\fi

\usepackage{amssymb, amsmath, amsthm, amsfonts}

\def\bibsection{\section*{References}} %%% Make "References" appear before bibliography


\usepackage[round]{natbib}
\bibliographystyle{plainnat}

\usepackage{longtable}
\usepackage[margin=2.3cm,bottom=2cm,top=2.5cm, includefoot]{geometry}
\usepackage{fancyhdr}
\usepackage[bottom, hang, flushmargin]{footmisc}
\usepackage{graphicx}
\numberwithin{equation}{section}
\numberwithin{figure}{section}
\numberwithin{table}{section}
\setlength{\parindent}{0cm}
\setlength{\parskip}{1.3ex plus 0.5ex minus 0.3ex}
\usepackage{textcomp}
\renewcommand{\headrulewidth}{0.2pt}
\renewcommand{\footrulewidth}{0.3pt}

\usepackage{array}
\newcolumntype{x}[1]{>{\centering\arraybackslash\hspace{0pt}}p{#1}}

%%%%  Remove the "preprint submitted to" part. Don't worry about this either, it just looks better without it:
\makeatletter
\def\ps@pprintTitle{%
  \let\@oddhead\@empty
  \let\@evenhead\@empty
  \let\@oddfoot\@empty
  \let\@evenfoot\@oddfoot
}
\makeatother

 \def\tightlist{} % This allows for subbullets!

\usepackage{hyperref}
\hypersetup{breaklinks=true,
            bookmarks=true,
            colorlinks=true,
            citecolor=blue,
            urlcolor=blue,
            linkcolor=blue,
            pdfborder={0 0 0}}


% The following packages allow huxtable to work:
\usepackage{siunitx}
\usepackage{multirow}
\usepackage{hhline}
\usepackage{calc}
\usepackage{tabularx}
\usepackage{booktabs}
\usepackage{caption}
\usepackage{colortbl}

\urlstyle{same}  % don't use monospace font for urls
\setlength{\parindent}{0pt}
\setlength{\parskip}{6pt plus 2pt minus 1pt}
\setlength{\emergencystretch}{3em}  % prevent overfull lines
\setcounter{secnumdepth}{5}

%%% Use protect on footnotes to avoid problems with footnotes in titles
\let\rmarkdownfootnote\footnote%
\def\footnote{\protect\rmarkdownfootnote}
\IfFileExists{upquote.sty}{\usepackage{upquote}}{}

%%% Include extra packages specified by user
% Insert custom packages here as follows
% \usepackage{tikz}

%%% Hard setting column skips for reports - this ensures greater consistency and control over the length settings in the document.
%% page layout
%% paragraphs
\setlength{\baselineskip}{12pt plus 0pt minus 0pt}
\setlength{\parskip}{12pt plus 0pt minus 0pt}
\setlength{\parindent}{0pt plus 0pt minus 0pt}
%% floats
\setlength{\floatsep}{12pt plus 0 pt minus 0pt}
\setlength{\textfloatsep}{20pt plus 0pt minus 0pt}
\setlength{\intextsep}{14pt plus 0pt minus 0pt}
\setlength{\dbltextfloatsep}{20pt plus 0pt minus 0pt}
\setlength{\dblfloatsep}{14pt plus 0pt minus 0pt}
%% maths
\setlength{\abovedisplayskip}{12pt plus 0pt minus 0pt}
\setlength{\belowdisplayskip}{12pt plus 0pt minus 0pt}
%% lists
\setlength{\topsep}{10pt plus 0pt minus 0pt}
\setlength{\partopsep}{3pt plus 0pt minus 0pt}
\setlength{\itemsep}{5pt plus 0pt minus 0pt}
\setlength{\labelsep}{8mm plus 0mm minus 0mm}
\setlength{\parsep}{\the\parskip}
\setlength{\listparindent}{\the\parindent}
%% verbatim
\setlength{\fboxsep}{5pt plus 0pt minus 0pt}



\begin{document}

\begin{frontmatter}  %

\title{Working in Texevier example}

% Set to FALSE if wanting to remove title (for submission)




\author[Add1]{Roan Minnie}
\ead{roanminnie@gmail.com}





\address[Add1]{Stellenbosch University, Stellenbosch, South Africa}

\cortext[cor]{Corresponding author: Roan Minnie}


\vspace{1cm}

\begin{keyword}
\footnotesize{
Multivariate GARCH \sep Kalman Filter \sep Copula \\ \vspace{0.3cm}
\textit{JEL classification} L250 \sep L100
}
\end{keyword}
\vspace{0.5cm}
\end{frontmatter}



%________________________
% Header and Footers
%%%%%%%%%%%%%%%%%%%%%%%%%%%%%%%%%
\pagestyle{fancy}
\chead{}
\rhead{}
\lfoot{}
\rfoot{\footnotesize Page \thepage\\}
\lhead{}
%\rfoot{\footnotesize Page \thepage\ } % "e.g. Page 2"
\cfoot{}

%\setlength\headheight{30pt}
%%%%%%%%%%%%%%%%%%%%%%%%%%%%%%%%%
%________________________

\headsep 35pt % So that header does not go over title




\section{\texorpdfstring{Introduction
\label{Introduction}}{Introduction }}\label{introduction}

This is a short assignment to demonstrate that I am able to neatly write
up a summary that includes figures and tables. This article was writen
using Texevier (Katzke \protect\hyperlink{ref-Texevier}{2017})

\section{\texorpdfstring{Summary table
\label{summary}}{Summary table }}\label{summary-table}

\begin{table}[H]
\centering
\begin{tabular}{rlrrrrrr}
  \hline
 & Ticker & mean & std\_dev & mean.1 & std\_dev.1 & mean.2 & std\_dev.2 \\ 
  \hline
1 & JSE.ABSP.Close & -0.00 & 0.01 & 0.00 & 0.01 & -0.00 & 0.01 \\ 
  2 & JSE.BVT.Close & 0.00 & 0.02 & 0.00 & 0.01 & 0.00 & 0.01 \\ 
  3 & JSE.FSR.Close & 0.00 & 0.03 & 0.00 & 0.02 & 0.00 & 0.01 \\ 
  4 & JSE.NBKP.Close & -0.00 & 0.01 & 0.00 & 0.01 & -0.00 & 0.01 \\ 
  5 & JSE.RMH.Close & 0.00 & 0.03 & 0.00 & 0.02 & 0.00 & 0.02 \\ 
  6 & JSE.SBK.Close & 0.00 & 0.02 & 0.00 & 0.01 & 0.00 & 0.01 \\ 
  7 & JSE.SLM.Close & 0.00 & 0.02 & 0.00 & 0.01 & 0.00 & 0.02 \\ 
   \hline
\end{tabular}
\caption{Short Table Example \label{tab1}} 
\end{table}

From table \ref{tab1} we can see that it confirms the argument presented
in Tsay (\protect\hyperlink{ref-Tsay1989}{1989})\ldots{} Table
\ref{tab1} shows that the mean and standard deviation do not vary across
subsamples. \#Unconditional correlation

Table \ref{tab2} below shows the unconditional correlation of each of
the seven stocks.

\begingroup\fontsize{12pt}{13pt}\selectfont

\begin{longtable}{rrrrr}
  \toprule
 & Correlation & p-value & Lower CI & Upper CI \\ 
  \hline 
\endhead 
\hline 
{\footnotesize Continued on next page} 
\endfoot 
\endlastfoot 
 \midrule
JSE.ABSP.Close to JSE.ABSP.Close & 1.00 & 0.00 & 1.00 & 1.00 \\ 
  JSE.ABSP.Close to JSE.BVT.Close & 0.02 & 0.36 & -0.02 & 0.06 \\ 
  JSE.ABSP.Close to JSE.FSR.Close & 0.01 & 0.69 & -0.03 & 0.05 \\ 
  JSE.ABSP.Close to JSE.NBKP.Close & 0.18 & 0.00 & 0.14 & 0.22 \\ 
  JSE.ABSP.Close to JSE.RMH.Close & 0.05 & 0.03 & 0.00 & 0.09 \\ 
  JSE.ABSP.Close to JSE.SBK.Close & 0.04 & 0.03 & 0.00 & 0.09 \\ 
  JSE.ABSP.Close to JSE.SLM.Close & 0.04 & 0.07 & -0.00 & 0.08 \\ 
  JSE.BVT.Close to JSE.ABSP.Close & 0.02 & 0.36 & -0.02 & 0.06 \\ 
  JSE.BVT.Close to JSE.BVT.Close & 1.00 & 0.00 & 1.00 & 1.00 \\ 
  JSE.BVT.Close to JSE.FSR.Close & 0.50 & 0.00 & 0.47 & 0.53 \\ 
  JSE.BVT.Close to JSE.NBKP.Close & 0.04 & 0.06 & -0.00 & 0.08 \\ 
  JSE.BVT.Close to JSE.RMH.Close & 0.48 & 0.00 & 0.44 & 0.51 \\ 
  JSE.BVT.Close to JSE.SBK.Close & 0.50 & 0.00 & 0.47 & 0.53 \\ 
  JSE.BVT.Close to JSE.SLM.Close & 0.49 & 0.00 & 0.45 & 0.52 \\ 
  JSE.FSR.Close to JSE.ABSP.Close & 0.01 & 0.69 & -0.03 & 0.05 \\ 
  JSE.FSR.Close to JSE.BVT.Close & 0.50 & 0.00 & 0.47 & 0.53 \\ 
  JSE.FSR.Close to JSE.FSR.Close & 1.00 & 0.00 & 1.00 & 1.00 \\ 
  JSE.FSR.Close to JSE.NBKP.Close & 0.01 & 0.62 & -0.03 & 0.05 \\ 
  JSE.FSR.Close to JSE.RMH.Close & 0.76 & 0.00 & 0.74 & 0.78 \\ 
  JSE.FSR.Close to JSE.SBK.Close & 0.71 & 0.00 & 0.69 & 0.73 \\ 
  JSE.FSR.Close to JSE.SLM.Close & 0.51 & 0.00 & 0.48 & 0.54 \\ 
  JSE.NBKP.Close to JSE.ABSP.Close & 0.18 & 0.00 & 0.14 & 0.22 \\ 
  JSE.NBKP.Close to JSE.BVT.Close & 0.04 & 0.06 & -0.00 & 0.08 \\ 
  JSE.NBKP.Close to JSE.FSR.Close & 0.01 & 0.62 & -0.03 & 0.05 \\ 
  JSE.NBKP.Close to JSE.NBKP.Close & 1.00 & 0.00 & 1.00 & 1.00 \\ 
  JSE.NBKP.Close to JSE.RMH.Close & -0.00 & 0.90 & -0.04 & 0.04 \\ 
  JSE.NBKP.Close to JSE.SBK.Close & 0.02 & 0.31 & -0.02 & 0.06 \\ 
  JSE.NBKP.Close to JSE.SLM.Close & 0.04 & 0.05 & 0.00 & 0.08 \\ 
  JSE.RMH.Close to JSE.ABSP.Close & 0.05 & 0.03 & 0.00 & 0.09 \\ 
  JSE.RMH.Close to JSE.BVT.Close & 0.48 & 0.00 & 0.44 & 0.51 \\ 
  JSE.RMH.Close to JSE.FSR.Close & 0.76 & 0.00 & 0.74 & 0.78 \\ 
  JSE.RMH.Close to JSE.NBKP.Close & -0.00 & 0.90 & -0.04 & 0.04 \\ 
  JSE.RMH.Close to JSE.RMH.Close & 1.00 & 0.00 & 1.00 & 1.00 \\ 
  JSE.RMH.Close to JSE.SBK.Close & 0.65 & 0.00 & 0.63 & 0.67 \\ 
  JSE.RMH.Close to JSE.SLM.Close & 0.50 & 0.00 & 0.46 & 0.53 \\ 
  JSE.SBK.Close to JSE.ABSP.Close & 0.04 & 0.03 & 0.00 & 0.09 \\ 
  JSE.SBK.Close to JSE.BVT.Close & 0.50 & 0.00 & 0.47 & 0.53 \\ 
  JSE.SBK.Close to JSE.FSR.Close & 0.71 & 0.00 & 0.69 & 0.73 \\ 
  JSE.SBK.Close to JSE.NBKP.Close & 0.02 & 0.31 & -0.02 & 0.06 \\ 
  JSE.SBK.Close to JSE.RMH.Close & 0.65 & 0.00 & 0.63 & 0.67 \\ 
  JSE.SBK.Close to JSE.SBK.Close & 1.00 & 0.00 & 1.00 & 1.00 \\ 
  JSE.SBK.Close to JSE.SLM.Close & 0.52 & 0.00 & 0.49 & 0.55 \\ 
  JSE.SLM.Close to JSE.ABSP.Close & 0.04 & 0.07 & -0.00 & 0.08 \\ 
  JSE.SLM.Close to JSE.BVT.Close & 0.49 & 0.00 & 0.45 & 0.52 \\ 
  JSE.SLM.Close to JSE.FSR.Close & 0.51 & 0.00 & 0.48 & 0.54 \\ 
  JSE.SLM.Close to JSE.NBKP.Close & 0.04 & 0.05 & 0.00 & 0.08 \\ 
  JSE.SLM.Close to JSE.RMH.Close & 0.50 & 0.00 & 0.46 & 0.53 \\ 
  JSE.SLM.Close to JSE.SBK.Close & 0.52 & 0.00 & 0.49 & 0.55 \\ 
  JSE.SLM.Close to JSE.SLM.Close & 1.00 & 0.00 & 1.00 & 1.00 \\ 
   \bottomrule
\caption{Unconditional correlation between stocks \label{tab2}} 
\end{longtable}

\endgroup

\section{Plotting the arch processes}\label{plotting-the-arch-processes}

\begin{figure}[H]

{\centering \includegraphics{Template_files/figure-latex/figure2-1} 

}

\caption{ABSP \label{absp}}\label{fig:figure2}
\end{figure}\begin{figure}[H]

{\centering \includegraphics{Template_files/figure-latex/figure3-1} 

}

\caption{BVT \label{bvt}}\label{fig:figure3}
\end{figure}\begin{figure}[H]

{\centering \includegraphics{Template_files/figure-latex/figure4-1} 

}

\caption{FSR \label{fsr}}\label{fig:figure4}
\end{figure}\begin{figure}[H]

{\centering \includegraphics{Template_files/figure-latex/figure5-1} 

}

\caption{NBKP \label{nbkp}}\label{fig:figure5}
\end{figure}\begin{figure}[H]

{\centering \includegraphics{Template_files/figure-latex/figure6-1} 

}

\caption{RMH \label{rmh}}\label{fig:figure6}
\end{figure}\begin{figure}[H]

{\centering \includegraphics{Template_files/figure-latex/figure7-1} 

}

\caption{SBK \label{sbk}}\label{fig:figure7}
\end{figure}\begin{figure}[H]

{\centering \includegraphics{Template_files/figure-latex/figure8-1} 

}

\caption{SLM \label{slm}}\label{fig:figure8}
\end{figure}

\section{Cumulative returns for a
portfolio}\label{cumulative-returns-for-a-portfolio}

Below is the cumulative returns series of a portfolio that is equally
weighted to each of the stocks, reweighted each year on the last day of
June.

\begin{figure}[H]

{\centering \includegraphics{Template_files/figure-latex/figure9-1} 

}

\caption{Portfolio cumulative returns \label{sbk}}\label{fig:figure9}
\end{figure}

\section*{References}\label{references}
\addcontentsline{toc}{section}{References}

\hypertarget{refs}{}
\hypertarget{ref-Texevier}{}
Katzke, N.F. 2017. \emph{Texevier: Package to Create Elsevier Templates
for Rmarkdown}. Stellenbosch, South Africa: Bureau for Economic
Research.

\hypertarget{ref-Tsay1989}{}
Tsay, Ruey S. 1989. ``Testing and Modeling Threshold Autoregressive
Processes.'' \emph{Journal of the American Statistical Association} 84
(405). Taylor \& Francis Group: 231--40.

% Force include bibliography in my chosen format:

\bibliographystyle{Tex/Texevier}
\bibliography{Tex/ref}





\end{document}
